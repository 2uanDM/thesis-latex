\documentclass[../main.tex]{subfiles}
\begin{document}

This thesis closes the critical gap between advanced person Re-ID technology and its practical, real-world application. Through the development of a novel hybrid edge-server architecture, the research demonstrates that sophisticated Re-ID capabilities can be deployed cost-effectively on resource-constrained devices. This breakthrough removes the traditional cost and infrastructure barriers that have historically limited the widespread adoption of this technology.

\section{Achievements}

The primary contributions of this research encompass both theoretical innovations and practical implementations that collectively advance the field of accessible person Re-ID systems.

\textbf{System architecture and design}

The thesis successfully developed and implemented a comprehensive hybrid edge-server person Re-ID system that optimally distributes computational workload between lightweight CPU-based edge devices and a centralized GPU-enabled server. This architecture addresses the fundamental limitations of existing approaches by:

\begin{itemize}
    \item Eliminating the need for expensive GPU hardware at edge locations while maintaining real-time processing capabilities
    \item Implementing a scalable microservices architecture using Docker containerization that enables horizontal scaling
    \item Integrating Apache Kafka message broker cluster for reliable, asynchronous communication between edge devices and the central server
    \item Achieving 99.5\% system uptime through robust fault-tolerance mechanisms
\end{itemize}

\textbf{CPU-based edge device optimization}

The research achieved significant breakthroughs in optimizing person detection for resource-constrained environments. The edge device implementation demonstrates:

\begin{itemize}
    \item Successful deployment of YOLOv11n model achieving 39.5\% mAP at 12 FPS on minimal hardware (1 CPU core at 3.5 GHz, 512MB RAM)
    \item Proof-of-concept that GPU acceleration is not requisite for effective person Re-ID deployment
\end{itemize}

\textbf{Metadata-enhanced retrieval innovation}

A key innovation of this thesis is the integration of lightweight gender classification for metadata enhancement, resulting in:

\begin{itemize}
    \item Development of a custom-trained gender classification model achieving 95\% accuracy using EfficientNet-B0 architecture
    \item Reduced computational overhead on the centralized server during peak operations
\end{itemize}

\textbf{Scalable Model Serving Framework}

The implementation of Ray Serve for model inference services achieved remarkable performance metrics:

\begin{itemize}
    \item Up to 121 RPS throughput for model inferencing through optimized batch processing
    \item 20\% performance improvement over traditional inference methods (82.30 vs 68.58 RPS)
    \item 176.4\% improvement in concurrent processing scenarios, making the system ideal for high-throughput applications
    \item Efficient dynamic batching mechanisms that optimize GPU utilization while maintaining low latency
\end{itemize}

\section{Limitations}

While this research achieves significant advances in accessible person Re-ID technology, several limitations must be acknowledged to provide a balanced perspective on the current state of the system.

\textbf{Hardware constraints and performance trade-offs}

\begin{itemize}
    \item The 12 FPS processing rate on edge devices may not be sufficient for high-speed surveillance scenarios or applications requiring real-time response
    \item Ray Serve achieves good performance with 121 RPS throughput and 176.4\% improvement in concurrent processing, but performance degrades severely (-80.9\%) under burst load patterns, indicating limitations in handling sudden traffic spikes
\end{itemize}

\section{Suggestions for future work}

\begin{itemize}
    \item \textbf{Handling burst requests performance degradation problem}: The current Ray Serve implementation experiences severe performance degradation (-80.9\%) under burst load patterns. Future work should investigate adaptive load balancing strategies, dynamic resource allocation mechanisms, and queue management techniques to maintain consistent performance during traffic spikes. Implementing circuit breaker patterns and auto-scaling policies could help mitigate this limitation.
    
    \item \textbf{Trying more detection models on limited resource CPU-based edge device to get better FPS}: While YOLOv11n achieves 12 FPS on minimal hardware, exploring alternative lightweight detection models such as MobileNet-SSD, EfficientDet, or newer YOLO variants could potentially improve frame rates. Additionally, investigating model quantization techniques, pruning strategies, and hardware-specific optimizations (e.g., Intel OpenVINO) may further enhance performance on CPU-constrained edge devices.
    
    \item \textbf{Trying another tracking algorithm like BOTSort, utilizing features to enhance the tracking performance}: The current system could benefit from implementing advanced multi-object tracking algorithms such as BOTSort\cite{aharon2022botsortrobustassociationsmultipedestrian}. These algorithms leverage appearance features and motion prediction to improve tracking accuracy and reduce identity switches, which would enhance the overall Re-ID system performance, especially in crowded scenarios with frequent occlusions.
\end{itemize}

\end{document}