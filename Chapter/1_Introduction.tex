\documentclass[../main.tex]{subfiles}
\begin{document}

This chapter explores the significant financial constraints that prevent many organizations from implementing advanced solutions to enhance user experience, particularly those based on AI. This challenge forms the foundational motivation for this thesis: to develop a solution that enables organizations with limited resources to access and benefit from powerful technologies like person Re-ID.

\section{Background and motivation}
\label{sec:background}

\subsection{Background}

In today's competitive market, many businesses face an extraordinary challenge in meeting evolving customer expectations, a hurdle felt most acutely by those operating on tight budgets. Customer expectations have fundamentally shifted from simple product transactions to demanding high-quality service provided by staff members. This change is particularly evident in sectors like \gls{FnB} and retail, where 65\% of customers report that staff responsiveness and personalized attention influence their purchasing decisions more than traditional advertising \cite{customer_experience}.

These financial constraints are a defining operational reality for many organizations, setting them apart from large corporations. In the FnB sector alone, 45\% of businesses report that raw materials account for over 30\% of their selling prices, leaving little room for major technology investments \cite{customer_experience2}. Furthermore, over 60\% of FnB businesses have experienced revenue decreases while facing rising operational costs, including rent, labor, and materials \cite{customer_experience3}.

This creates an ``innovation deadlock'' where these organizations:
\begin{itemize}
    \item Recognize the critical need for better staff management and service optimization solutions.
    \item Understand that technology could provide competitive advantages in monitoring and improving staff performance, yet lack the capital to invest.
\end{itemize}

To help organizations with limited hardware resources still access Re-ID technology, they need solutions for surveillance problems like staff Re-ID to visualize activity maps and optimize service delivery. Modern AI technologies like Re-ID offer powerful solutions for understanding staff behavior patterns, monitoring service coverage across different areas, and identifying bottlenecks in staff allocation. Re-ID systems can seamlessly track staff movements across different sections of a store or restaurant, measure response times to customer needs, and identify areas where additional staff presence is required.

However, traditional Re-ID systems present significant economic barriers. Conventional implementations require expensive GPU-powered edge devices. When scaled across multiple cameras needed for comprehensive coverage, these costs become prohibitive. The high computational requirements of traditional Re-ID systems also demand powerful central servers for data processing and storage, further inflating the total cost of ownership. For organizations already struggling with thin profit margins, these substantial upfront investments often exceed their entire annual technology budgets.

\subsection{Motivation}
\label{sec:motivation}

The challenges outlined above highlight a critical gap in the market: while Re-ID technology offers tremendous potential for staff monitoring and operational optimization, existing solutions remain financially inaccessible to organizations with limited resources. This creates an urgent need for affordable, scalable Re-ID solutions designed specifically for environments with limited hardware resources.

To address this need effectively, it becomes essential to examine existing architectural approaches and understand why they fall short for resource-constrained environments. The architectural design of a person Re-ID system is a critical consideration, with each approach presenting distinct trade-offs. The chosen architecture directly impacts the system's cost, scalability, and real-time performance, making it an especially important factor for organizations operating under budget constraints. The primary architectural models are \textbf{fully centralized}, \textbf{fully edge-based} and \textbf{hybrid}, yet none of these fully address the unique challenges faced by organizations.

To understand why current solutions fall short for organizations with limited resources, it is essential to examine each architectural approach and its specific limitations. The following subsections analyze these three primary models in detail:

\subsubsection{Fully centralized architecture}
The conventional approach involves a centralized architecture where standard cameras transmit their video feeds over a network to a single, powerful server. This central server is responsible for all computationally demanding tasks, such as person detection, feature extraction, and identity matching.

While this model simplifies on-site hardware, it introduces significant drawbacks that render it impractical for organizations with limited resources:
\begin{itemize}
    \item \textbf{Network dependency and costs:} The continuous streaming of video from multiple cameras requires substantial network bandwidth, leading to high operational costs. The system's reliability is also contingent on network stability, as latency or packet loss can result in incomplete data.
    \item \textbf{High server costs:} The computational load on the central server scales with the number of cameras, necessitating a significant upfront investment in high-performance server hardware. For many companies, this cost is prohibitive. This architecture also creates a single point of failure.
    \item \textbf{Management complexity:} A Re-ID pipeline consists of multiple processing stages. Managing this complex workflow for numerous concurrent video streams on a single machine presents a considerable technical challenge.
\end{itemize}

\subsubsection{Fully edge-based architecture}
To mitigate the network dependencies of the centralized model, an edge-based architecture places computational power on devices located near the cameras. These edge evices process video locally and transmit only lightweight metadata, such as feature vectors, to a central location.

This design reduces network bandwidth requirements and enhances resilience to network disruptions. However, it introduces its own distinct disadvantages:
\begin{itemize}
    \item \textbf{High cumulative hardware cost:} The primary issue is the cost of the edge devices. While a single unit may be affordable, the expense escalates with each camera added, making large-scale deployments costly.
    \item \textbf{Distributed maintenance:} Managing a distributed fleet of edge devices is operationally more complex than maintaining a single server, increasing the burden of software updates and hardware troubleshooting.
\end{itemize}

\subsubsection{Hybrid architecture}
A hybrid architecture attempts to strike a balance by distributing the workload between edge devices and a central server. For instance, a low-cost edge device might handle initial person detection, while the more intensive matching tasks are offloaded to the server.

This approach aims to reduce hardware costs at the edge, but it presents its own set of complexities:
\begin{itemize}
    \item \textbf{System integration challenges:} Dividing tasks creates a more intricate, multi-tiered system. Ensuring seamless communication and efficient integration between the edge and server components is a significant engineering task.
    \item \textbf{Potential for latency:} The handoff of data between the edge and the server can introduce processing delays. In applications requiring immediate responses, this latency can undermine the system's utility.
    \item \textbf{Workload balancing:} Achieving an optimal balance is difficult. If the edge device is underpowered, it can become a bottleneck. Conversely, if too much processing is offloaded to the server, the architecture reintroduces the bandwidth and cost issues of the centralized model.
\end{itemize}

In summary, the analysis of existing Re-ID architectures reveals a significant opportunity: while current solutions fail to meet the needs of resource-constrained organizations due to prohibitive costs, network dependencies, and operational complexity, the underlying technology holds immense potential for transforming staff monitoring and operational efficiency. This gap between technological capability and practical accessibility represents not just a challenge, but a compelling opportunity to democratize advanced surveillance technology.

The persistent ``innovation deadlock'' that prevents smaller organizations from leveraging Re-ID technology can be broken through thoughtful engineering and architectural innovation. By addressing the fundamental cost and complexity barriers, we can unlock the transformative potential of person Re-ID for organizations that need it most-those operating with limited resources but requiring maximum operational efficiency.

This thesis presents a feasible solution: a novel hybrid architecture specifically engineered to make Re-ID technology affordable, scalable, and manageable for surveillance systems in resource-constrained environments. Our approach promises to bridge the gap between cutting-edge AI capabilities and practical deployment constraints, enabling organizations of all sizes to harness the power of intelligent staff monitoring and operational optimization.

\section{Objectives and scope of thesis} 
\label{sec:objectives}

Section \ref{sec:background} presents various challenges that small organizations are facing due to their limited financial ability to implement solutions for internal surveillance. The first objective of this thesis is to design and implement a hybrid edge-server person Re-ID pipeline that can be deployed on resource-constrained devices, especially on CPU-based devices for edge deployment. This proves that GPU-based devices are not requisite for person Re-ID. This pipeline can detect, extract semantic vision features, and track person movements to report the results to the central server. The output of the pipeline can be used for various applications such as staff monitoring, service coverage analysis, and operational optimization.

Furthermore, to enhance the speed and accuracy of identity retrieval, a specialized gender classification model is proposed to be deployed on the edge device as a metadata enhancement component. This lightweight model is trained on a carefully prepared dataset and achieves high classification accuracy while maintaining\\ computational efficiency suitable. By incorporating gender information as metadata, the system can intelligently partition the search space during identity matching operations, significantly reducing computational overhead and improving retrieval performance. This metadata-driven approach not only accelerates the matching process but also enhances the overall system's scalability by reducing the workload on the central server during peak operations.

The scope of this thesis is limited to implementing a complete pipeline using Docker containerization, with resource constraints of 1 CPU core and 512MB RAM per edge device simulation. This configuration enables the creation of multiple virtual edge devices on a single physical machine for comprehensive testing and validation. Data collected from these simulated edge devices is transmitted to the centralized server through a Kafka message broker cluster, ensuring reliable and scalable communication. Both the Kafka cluster and centralized server are implemented as Docker containers, providing a deployment-ready solution that can be easily adapted for production environments with minimal network configuration adjustments.


\section{Tentative solution} 
\label{sec:tentative_solution}

Based on the presented objectives in Section \ref{sec:objectives}, the work composes of two main parts.

In the first part, based on researched technologies, the pipeline of the system will be proposed. The system will consist of three main components: edge devices with CPU-only computing capabilities, Kafka message broker cluster, and a centralized server. The edge devices will be responsible for human detection and basic preprocessing before sending data to Kafka message brokers. These lightweight edge devices eliminate the need for expensive GPU hardware while maintaining real-time processing capabilities. The Kafka message broker cluster serves as the communication backbone, ensuring reliable and scalable message delivery between edge devices and the central server. The third component is the centralized server, which combines multiple services following a microservices architecture including Model service, ID verifier service, vector database, and multiple consumers that perform feature extraction and identity matching using these services. The human detection model will be carefully selected among state-of-the-art models based on the criteria of accuracy and computational efficiency so that it can run effectively on CPU-only edge devices. This distributed architecture ensures optimal resource utilization while maintaining system scalability and cost-effectiveness.

The second part includes building a tailored, lightweight gender classification model specifically for limited computing resources and integrating it into the Re-ID pipeline for metadata enhancement. A custom gender classification model will be developed using EfficientNet-B0 architecture and trained on the P-DESTRE dataset to achieve high accuracy while maintaining computational efficiency. This gender classification capability will be integrated into the system to enhance the retrieval process by reducing the search space during identity matching operations. A prototype will be deployed in practice for comprehensive testing purposes, validating the system's performance in real-world scenarios and demonstrating its viability for deploying on resource-constrained devices.


\section{Contributions}
\label{sec:contribution}

\begin{enumerate}
    \item An application is deployed on hybrid edge-server devices and uses a microservices architecture, allowing for easy system scaling (increasing the number of cameras). It includes:
    \begin{itemize}
        \item An optimized Docker image for running the edge device processing flow with custom CPU and RAM configurations, tailored for deployment on CPU-only devices.
        \item Optimized Docker images for deploying multiple model inference services through Ray Serve, including feature extraction and gender classification models, enabling high throughput and low latency for model inference via RESTful APIs.
        \item A custom-trained, lightweight gender classification model with 95\% accuracy for metadata enhancement.
        \item A vector database optimization algorithm for efficient identity retrieval. This uses a person's metadata (gender) to reduce the search space, improving retrieval speed and accuracy.
    \end{itemize} 
\item This thesis also provides an interactive web application. It lets users monitor the system, view live camera streams.
\end{enumerate}

\section{Organization of thesis}
\label{sec:organize}

The remaining of this thesis will be organized as follows:

\textbf{Chapter 2} presents a comprehensive literature review and foundation theory essential for understanding the proposed system. The chapter is divided into two main sections: related works covering recent advances in person Re-ID, edge computing in AI, and microservices architectures; and foundation theory providing detailed technical background on object detection (YOLOv11), object tracking (ByteTrack), feature extraction (OSNet and LightMBN), image classification (EfficientNet-B0), message queuing (Apache Kafka), model serving frameworks (Ray Serve), containerization (Docker), and vector databases (Redis). These investigations form the theoretical foundation upon which the proposed hybrid edge-server system is built.

\textbf{Chapter 3} introduces the comprehensive methodology of the proposed hybrid edge-server person Re-ID system. This chapter presents the system overview and detailed implementation of three main components: the Kafka message broker cluster for reliable asynchronous communication, edge devices optimized for CPU-only human detection and preprocessing, and the centralized server managing microservices including model inference, vector database operations, and identity matching. The chapter also covers the deployment strategies, hardware implementations, and the integration of lightweight models specifically designed for resource-constrained environments.

\textbf{Chapter 4} presents the experimental results and evaluation of the deployed system. This chapter covers the environmental setup including hardware specifications for both edge devices and the centralized server, dataset preparation and evaluation metrics, performance analysis of edge device processing capabilities, and evaluation of the custom gender classification model. The experimental results demonstrate the system's effectiveness in real-world scenarios and validate the viability of the proposed approach for deployment on resource-constrained devices.

\textbf{Chapter 5}, the final chapter, provides a comprehensive summary of the achieved results, discusses the contributions and limitations of the proposed system, and presents suggestions for future improvements and research directions.


\end{document}