\documentclass[../main.tex]{subfiles}
\begin{document}

This chapter provides the understanding of the current circumstances that SMEs are facing due to their limited financial ability to implement solutions for enhancing user experience, particularly using AI solutions. This also forms the foundation that brought me to develop this thesis, helping SMEs access person Re-ID technologies with limited budgets.


\section{Background and Motivation}
\label{sec:motivation}

Small and Medium Enterprises (SMEs) across various industries are facing an extraordinary challenge in today's competitive market. Customer expectations have fundamentally shifted from simple product transactions to demanding rich, personalized experiences. This change is particularly evident in sectors like Food \& Beverage (F\&B) and retail, where 65\% of customers report that positive experiences influence their purchasing decisions more than traditional advertising \cite{customer_experience}.

SMEs operate under much tighter financial constraints than large corporations. In the F\&B sector alone, 45\% of businesses report that raw materials account for over 30\% of their selling prices, leaving little room for major technology investments \cite{customer_experience2}. Over 60\% of F\&B businesses have experienced revenue decreases while facing rising operational costs including rent, labor, and materials \cite{customer_experience3}.

This creates an "innovation deadlock" where SMEs:
\begin{itemize}
    \item Recognize the critical need for better customer experience solutions.
    \item Understand that technology could provide competitive advantages.
\end{itemize}

Modern AI technologies like Re-ID offer powerful solutions for understanding customer behavior, optimizing store layouts, and creating personalized experiences. Re-ID systems can seamlessly track customer movements across different areas of a store, measure how long customers spend in specific sections, and identify popular pathways and bottlenecks. This technology enables businesses to provide tailored assistance, highlight relevant promotions based on customer interests, and optimize staff allocation in real-time.

However, traditional Re-ID systems present significant economic barriers that make them inaccessible to most SMEs. Conventional implementations require \\ expensive GPU-powered edge devices. When scaled across multiple cameras needed for comprehensive coverage, these costs become prohibitive.

The high computational requirements of traditional Re-ID systems also demand powerful central servers for data processing and storage, further inflating the total cost of ownership. For SMEs already struggling with thin profit margins, these substantial upfront investments often exceed their entire annual technology budgets.

Therefore, there is an urgent need for affordable, scalable Re-ID solutions designed specifically for SME use. Such systems should lower hardware costs by using efficient CPU-based processing at the edge, reduce complex setup requirements, and provide useful customer experience improvements that help SMEs compete on service quality rather than just price. By making intelligent customer interaction technologies accessible to all businesses, we can help companies of all sizes improve customer satisfaction, build loyalty, and achieve steady growth in today's experience-focused marketplace.

\section{Challenges and Current Solutions}
\label{sec:tentative}

The architectural design of a person Re-ID system is a critical consideration, with each approach presenting distinct trade-offs. The chosen architecture directly impacts the system's cost, scalability, and real-time performance, making it an especially important factor for SMEs operating under budget constraints. The primary architectural models are centralized, edge-based, and hybrid, yet none of these fully address the unique challenges faced by small businesses.

\subsection{Fully Centralized Architecture}
The conventional approach involves a centralized architecture where standard cameras transmit their video feeds over a network to a single, powerful server. This central server is responsible for all computationally demanding tasks, such as person detection, feature extraction, and identity matching.

While this model simplifies on-site hardware, it introduces significant drawbacks that render it impractical for most SMEs:
\begin{itemize}
    \item \textbf{Network Dependency and Costs:} The continuous streaming of video from multiple cameras requires substantial network bandwidth, leading to high operational costs. The system's reliability is also contingent on network stability, as latency or packet loss can result in incomplete data.
    \item \textbf{High Server Costs:} The computational load on the central server scales with the number of cameras, necessitating a significant upfront investment in high-performance server hardware. For many SMEs, this cost is prohibitive. This architecture also creates a single point of failure.
    \item \textbf{Management Complexity:} A Re-ID pipeline consists of multiple processing stages. Managing this complex workflow for numerous concurrent video streams on a single machine presents a considerable technical challenge.
\end{itemize}

\subsection{Fully Edge-Based Architecture}
To mitigate the network dependencies of the centralized model, an edge-based architecture places computational power on devices located near the cameras. These "edge" devices process video locally and transmit only lightweight metadata, such as feature vectors, to a central location.

This design reduces network bandwidth requirements and enhances resilience to network disruptions. However, it introduces its own distinct disadvantages for SMEs:
\begin{itemize}
    \item \textbf{High Cumulative Hardware Cost:} The primary issue is the cost of the edge devices. While a single unit may be affordable, the expense escalates with each camera added, making large-scale deployments costly.
    \item \textbf{Distributed Maintenance:} Managing a distributed fleet of edge devices is operationally more complex than maintaining a single server, increasing the burden of software updates and hardware troubleshooting.
\end{itemize}

\subsection{Hybrid Architecture}
A hybrid architecture attempts to strike a balance by distributing the workload between edge devices and a central server. For instance, a low-cost edge device might handle initial person detection, while the more intensive matching tasks are offloaded to the server.

This approach aims to reduce hardware costs at the edge, but it presents its own set of complexities:
\begin{itemize}
    \item \textbf{System Integration Challenges:} Dividing tasks creates a more intricate, multi-tiered system. Ensuring seamless communication and efficient integration between the edge and server components is a significant engineering task.
    \item \textbf{Potential for Latency:} The handoff of data between the edge and the server can introduce processing delays. In applications requiring immediate responses, this latency can undermine the system's utility.
    \item \textbf{Workload Balancing:} Achieving an optimal balance is difficult. If the edge device is underpowered, it can become a bottleneck. Conversely, if too much processing is offloaded to the server, the architecture reintroduces the bandwidth and cost issues of the centralized model.
\end{itemize}
In summary, current Re-ID architectures do not present ideal solutions for SMEs due to challenges related to cost, network dependency, and complexity. This "innovation deadlock" hinders smaller businesses from adopting this valuable technology. This thesis proposes a novel hybrid solution engineered to be affordable, scalable, and manageable within an SME context.

\section{Objectives and scope of Thesis} 
\label{sec:objectives}

This thesis aims to address the challenges faced by SMEs in adopting person Re-ID technology by developing a cost-effective, scalable solution that combines the benefits of edge computing with centralized processing power.

The primary objective is to design and implement an end-to-end person Re-ID pipeline that can operate efficiently on CPU-based edge devices without requiring expensive GPU acceleration. This approach significantly reduces hardware costs while maintaining acceptable performance levels for real-world applications.

The scope of this thesis encompasses several key areas:

\begin{itemize}
    \item \textbf{System Architecture Design:} Development of a hybrid edge-server architecture that balances computational load between lightweight edge devices and a central server. This design minimizes network bandwidth requirements while keeping hardware costs manageable for SMEs.
    
    \item \textbf{AI Model Development and Training:} Implementation of custom lightweight models optimized for CPU inference, including:
    \begin{itemize}
        \item A person detection model achieving 85\% mAP at 14 FPS on only 1 CPU core and 512 MB of RAM
        \item A gender classification model with 95\% accuracy for metadata enhancement
    \end{itemize}
    
    \item \textbf{Intelligent Retrieval Optimization:} Development of a metadata-enhanced search algorithm that uses gender classification results to reduce the search space during identity matching, improving retrieval speed by 40\% and accuracy by 8\% compared to traditional approaches.
    
    \item \textbf{Containerized Deployment:} Implementation of the entire system using \\ containerization technology to ensure easy deployment, consistent performance across different environments, and simplified maintenance procedures.
    
    \item \textbf{Microservices Implementation:} Breaking down the Re-ID pipeline into \\ independent microservices, particularly for AI model serving through HTTP APIs. This approach enables horizontal scaling, improves system reliability, and allows for independent updates of system components.

    \item \textbf{High-Throughput Model Serving:} Optimization of the inference pipeline to achieve up to 170 requests per second (RPS) for a 7 million parameter model. This includes full GPU utilization on the central server and elimination of bottlenecks that could limit processing throughput.
    
    \item \textbf{System Reliability:} Integration of monitoring, alerting, and health check mechanisms to ensure high system availability (99.5\% uptime) and quick identification of potential issues.
\end{itemize}

The ultimate goal is to provide SMEs with an accessible path to implement person Re-ID technology that fits within their budget constraints while delivering the customer experience improvements they need to remain competitive in today's market.


\section{Contributions}
\label{sec:contribution}

% This thesis presents two main contributions:
\begin{enumerate}
    \item An application is deployed on hybrid edge-server devices and uses a microservices architecture, allowing for easy system scaling (increasing the number of cameras). It includes:
    \begin{itemize}
        \item A custom-trained, lightweight gender classification model with 95\% accuracy for metadata enhancement.
        \item A vector database optimization algorithm for efficient identity retrieval. This uses a person's metadata (gender) to reduce the search space, improving retrieval speed and accuracy.
    \item This thesis also provides an interactive web application. It lets users monitor the system, view live camera streams, and search for people using their metadata.
\end{itemize} 
\end{enumerate}

\section{Organization of Thesis}
\label{sec:organize}


\end{document}