\documentclass[../main.tex]{subfiles}
\begin{document}

\begin{center}
    \Large{\textbf{ABSTRACT}}\\
\end{center}
\vspace{1cm}

SMEs face significant challenges in adopting person Re-ID technology due to high costs and complex infrastructure requirements. This thesis addresses these limitations by proposing a novel hybrid edge-server architecture specifically designed for cost-effective person Re-ID deployment in SME environments.

The proposed system distributes computational workload between lightweight CPU-based edge devices and a centralized GPU-enabled server, connected through an Apache Kafka message broker cluster. Edge devices perform human detection using an optimized YOLOv11n model, achieving 39.5\% mAP at 12 FPS on average on minimal hardware with 1 CPU core at 3.5 GHz and 512MB RAM. The centralized server handles computationally intensive tasks including feature extraction, gender classification, and identity matching across multiple cameras.

A key innovation is the integration of metadata-enhanced retrieval optimization, where a custom-trained gender classification model with 95\% accuracy reduces the search space during identity matching, improving retrieval speed by 40\% compared to traditional approaches. The system is implemented using containerized microservices architecture, enabling horizontal scaling and achieving up to 170 RPS for models inferencing with 99.5\% system uptime.

This research contributes a practical, scalable solution that makes person Re-ID technology accessible to organizations with limited resources, enabling them to access surveillance applications in staffs management, security monitoring and operational efficiency without prohibitive infrastructure investments. The hybrid architecture proves that intelligent distribution of computational tasks can overcome the traditional cost-performance trade-offs in person Re-ID systems.

\end{document}