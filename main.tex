\documentclass[a4paper,13pt,3p,oneside]{report}
\usepackage{scrextend}
\changefontsizes{13pt}
\usepackage[utf8]{vietnam}
\usepackage[top=2cm, bottom=2cm, left=3.5cm, right=2.5cm]{geometry}
\usepackage{xurl}
\usepackage{makecell}
\usepackage{xcolor}
\usepackage{appendix}
\usepackage{babel}
\usepackage{outlines}
\usepackage{graphicx} % Cho phép chèn hỉnh ảnh
\usepackage{fancybox} % Tạo khung box
\usepackage{indentfirst} % Thụt đầu dòng ở dòng đầu tiên trong đoạn
\usepackage{amsthm} % Cho phép thêm các môi trường định nghĩa
\usepackage{latexsym} % Các kí hiệu toán học
\usepackage{amsmath} % Hỗ trợ một số biểu thức toán học
\usepackage{amssymb} % Bổ sung thêm kí hiệu về toán học
\usepackage{amsbsy} % Hỗ trợ các kí hiệu in đậm
\usepackage{times} % Chọn font Time New Romans
\usepackage{array} % Tạo bảng array
\usepackage{enumitem} % Cho phép thay đổi kí hiệu của list
\usepackage{subfiles} % Chèn các file nhỏ, giúp chia các chapter ra nhiều file hơn
\usepackage{titlesec} % Giúp chỉnh sửa các tiêu đề, đề mục như chương, phần,..
\usepackage{titletoc}
\usepackage{chngcntr} % Dùng để thiết lập lại cách đánh số caption,..
\usepackage{pdflscape} % Đưa các bảng có kích thước đặt theo chiều ngang giấy
\usepackage{afterpage}
\usepackage[ruled,vlined]{algorithm2e}  % Hỗ trợ viết các giải thuật
\usepackage{capt-of} % Cho phép sử dụng caption lớn đối với landscape page
\usepackage{multirow} % Merge cells
\usepackage{fancyhdr} % Cho phép tùy biến header và footer
% \usepackage[natbib,backend=biber,style=ieee]{biblatex} % Giúp chèn tài liệu tham khảo

\usepackage[font=small,labelfont=bf]{caption}

\usepackage{listings}
\usepackage{float}
\usepackage{subcaption}

\usepackage[nonumberlist, nopostdot, nogroupskip, acronym]{glossaries}
\usepackage{glossary-superragged}
\setglossarystyle{superraggedheaderborder}
\usepackage{setspace}
\usepackage{parskip}

% package content table
\usepackage{tocbasic}

\usepackage{blindtext}

% Define colors for syntax highlighting
\definecolor{dkgreen}{rgb}{0,0.6,0}
\definecolor{darkgreen}{rgb}{0,0.5,0}
\definecolor{gray}{rgb}{0.5,0.5,0.5}
\definecolor{darkgray}{rgb}{0.3,0.3,0.3}
\definecolor{lightgray}{rgb}{0.95,0.95,0.95}
\definecolor{mauve}{rgb}{0.58,0,0.82}
\definecolor{orange}{rgb}{1,0.5,0}
\definecolor{teal}{rgb}{0,0.5,0.5}
\definecolor{darkred}{rgb}{0.5,0,0}
\definecolor{purple}{rgb}{0.5,0,0.5}


% ===================================================

% \renewcommand{\bibname}{Danh_sach_tai_lieu_tham_khao} 
\usepackage[backend=bibtex,style=ieee]{biblatex}  %backend=biber is 'better'

\addbibresource{reference.bib} % chèn file chứa danh mục tài liệu tham khảo vào 



%\makeglossaries
\makenoidxglossaries

% Danh mục thuật ngữ và từ viết tắt
\newglossaryentry{AI}{
    type=\acronymtype,
    name={AI},
    description={Artificial Intelligence},
    first={Artificial Intelligence}
}
\newglossaryentry{FPS}{
    type=\acronymtype,
    name={FPS},
    description={Frames Per Second},
    first={Frames Per Second}
}
\newglossaryentry{mAP}{
    type=\acronymtype,
    name={mAP},
    description={mean Average Precision},
    first={mean Average Precision}
}
\newglossaryentry{Re-ID}{
    type=\acronymtype,
    name={Re-ID},
    description={Re-identification},
    first={Re-identification}
}
\newglossaryentry{CNN}{
    type=\acronymtype,
    name={CNN},
    description={Convolutional Neural Networks},
    first={Convolutional Neural Networks}
}
\newglossaryentry{SME}{
    type=\acronymtype,
    name={SME},
    description={Small and Medium Enterprises},
    first={Small and Medium Enterprises}
}
\newglossaryentry{IoU}{
    type=\acronymtype,
    name={IoU},
    description={Intersection over Union},
    first={Intersection over Union}
}
\newglossaryentry{ID}{
    type=\acronymtype,
    name={ID},
    description={Identification},
    first={Identification}
}
\newglossaryentry{YOLO}{
    type=\acronymtype,
    name={YOLO},
    description={You Only Look Once},
    first={You Only Look Once}
}
\newglossaryentry{RPS}{
    type=\acronymtype,
    name={RPS},
    description={Requests Per Second},
    first={Requests Per Second}
}
\newglossaryentry{FnB}{
    type=\acronymtype,
    name={FnB},
    description={Food and Beverage},
    first={Food and Beverage}
}
\newglossaryentry{CPU}{
    type=\acronymtype,
    name={CPU},
    description={Central Processing Unit},
    first={Central Processing Unit}
}
\newglossaryentry{GPU}{
    type=\acronymtype,
    name={GPU},
    description={Graphics Processing Unit},
    first={Graphics Processing Unit}
}
\newglossaryentry{HTTP}{
    type=\acronymtype,
    name={HTTP},
    description={Hypertext Transfer Protocol},
    first={Hypertext Transfer Protocol}
}
\newglossaryentry{PoC}{
    type=\acronymtype,
    name={PoC},
    description={Proof of Concept},
    first={Proof of Concept}
}
\newglossaryentry{API}{
    type=\acronymtype,
    name={API},
    description={Application Programming Interface},
    first={Application Programming Interface}
}
\newglossaryentry{RAM}{
    type=\acronymtype,
    name={RAM},
    description={Random Access Memory},
    first={Random Access Memory}
}
\newglossaryentry{OLTP}{
    type=\acronymtype,
    name={OLTP},
    description={Online Transaction Processing},
    first={Online Transaction Processing}
}
\newglossaryentry{OLAP}{
    type=\acronymtype,
    name={OLAP},
    description={Online Analytical Processing},
    first={Online Analytical Processing}
}
\newglossaryentry{UAV}{
    type=\acronymtype,
    name={UAV},
    description={Unmanned Aerial Vehicle},
    first={Unmanned Aerial Vehicle}
}
\newglossaryentry{MOT}{
    type=\acronymtype,
    name={MOT},
    description={Multi-Object Tracking},
    first={Multi-Object Tracking}
}


% ===================================================


\fancypagestyle{plain}{%
\fancyhf{} % clear all header and footer fields
\fancyfoot[RO]{\thepage} %RO=right odd
\renewcommand{\headrulewidth}{0pt}
\renewcommand{\footrulewidth}{0pt}}

\setlength{\headheight}{10pt}

\def \TITLE{GRADUATION THESIS}
\def \AUTHOR{Đường Minh Quân}

% ===================================================
\titleformat{\chapter}[hang]{\centering\bfseries}{CHAPTER \thechapter.\ }{0pt}{}[]

\titleformat 
    {\chapter} % command
    [hang] % shape
    {\centering\bfseries} % format
    {CHAPTER \thechapter.\ } % label
    {0pt} %sep
    {} % before
    [] % after
\titlespacing*{\chapter}{0pt}{-20pt}{20pt}

\titleformat
    {\section} % command
    [hang] % shape
    {\bfseries} % format
    {\thechapter.\arabic{section}\ \ \ \ } % label
    {0pt} %sep
    {} % before
    [] % after
\titlespacing{\section}{0pt}{\parskip}{0.5\parskip}

\titleformat
    {\subsection} % command
    [hang] % shape
    {\bfseries} % format
    {\thechapter.\arabic{section}.\arabic{subsection}\ \ \ \ } % label
    {0pt} %sep
    {} % before
    [] % after
\titlespacing{\subsection}{30pt}{\parskip}{0.5\parskip}

\renewcommand\thesubsubsection{\alph{subsubsection}}
\titleformat
    {\subsubsection} % command
    [hang] % shape
    {\bfseries} % format
    {\alph{subsubsection}, \ } % label
    {0pt} %sep
    {} % before
    [] % after
\titlespacing{\subsubsection}{50pt}{\parskip}{0.5\parskip}

% \newcommand{\titlesize}{\fontsize{18pt}{23pt}\selectfont}
% \newcommand{\subtitlesize}{\fontsize{16pt}{21pt}\selectfont}
% \titleclass{\part}{top}
% \titleformat{\part}[display]
%   {\normalfont\huge\bfseries}{\centering}{20pt}{\Huge\centering}
% \titlespacing{\part}{0pt}{em}{1em}
% \titlespacing{\section}{0pt}{\parskip}{0.5\parskip}
% \titlespacing{\subsection}{0pt}{\parskip}{0.5\parskip}
% \titlespacing{\subsubsection}{0pt}{\parskip}{0.5\parskip}



% ===================================================
\usepackage{hyperref}
\hypersetup{pdfborder = {0 0 0}}
\hypersetup{pdftitle={\TITLE},
	pdfauthor={\AUTHOR}}
	
\usepackage[all]{hypcap} % Cho phép tham chiếu chính xác đến hình ảnh và bảng biểu

\graphicspath{{figures/}{../figures/}} % Thư mục chứa các hình ảnh

\counterwithin{figure}{chapter} % Đánh số hình ảnh kèm theo chapter. Ví dụ: Hình 1.1, 1.2,..

\title{\bf \TITLE}
\author{\AUTHOR}

\setcounter{secnumdepth}{3} % Cho phép subsubsection trong report
% \setcounter{tocdepth}{3} % Chèn subsubsection vào bảng mục lục

\theoremstyle{definition}
\newtheorem{example}{Example}[chapter] % Định nghĩa môi trường ví dụ

\onehalfspacing
\setlength{\parskip}{6pt}
\setlength{\parindent}{15pt}



% =========================== BODY ===============
\begin{document}

% Language definition for YAML with enhanced syntax highlighting
\lstdefinelanguage{yaml}{
  keywords=[1]{true,false,null,yes,no,on,off},
  keywords=[2]{proxy_location,http_options,grpc_options,logging_config,applications,deployments,ray_actor_options,user_config,runtime_env},
  keywords=[3]{name,route_prefix,import_path,host,port,encoding,log_level,logs_dir,enable_access_log,max_ongoing_requests,num_cpus,num_gpus},
  keywordstyle=[1]\color{blue}\bfseries,
  keywordstyle=[2]\color{teal}\bfseries,
  keywordstyle=[3]\color{darkgreen}\bfseries,
  identifierstyle=\color{black},
  sensitive=true,
  comment=[l]{\#},
  commentstyle=\color{gray}\itshape,
  stringstyle=\color{orange},
  morestring=[b]',
  morestring=[b]",
  numberstyle=\color{purple},
  frame=none,
  literate=
    {0}{{{\color{purple}0}}}1
    {1}{{{\color{purple}1}}}1
    {2}{{{\color{purple}2}}}1
    {3}{{{\color{purple}3}}}1
    {4}{{{\color{purple}4}}}1
    {5}{{{\color{purple}5}}}1
    {6}{{{\color{purple}6}}}1
    {7}{{{\color{purple}7}}}1
    {8}{{{\color{purple}8}}}1
    {9}{{{\color{purple}9}}}1
    {0.0001}{{{\color{purple}0.0001}}}6
    {0.1}{{{\color{purple}0.1}}}3
    {0.2}{{{\color{purple}0.2}}}3
    {32}{{{\color{purple}32}}}2
    {64}{{{\color{purple}64}}}2
    {256}{{{\color{purple}256}}}3
    {512}{{{\color{purple}512}}}3
    {8000}{{{\color{purple}8000}}}4
    {9000}{{{\color{purple}9000}}}4
    {:}{{{\color{darkred}:}}}1
    {-}{{{\color{darkred}-}}}1
}

% Language definition for JSON with enhanced syntax highlighting
\lstdefinelanguage{json}{
  keywords=[1]{true,false,null},
  keywordstyle=[1]\color{blue}\bfseries,
  identifierstyle=\color{black},
  sensitive=true,
  comment=[l]{//},
  morecomment=[s]{/*}{*/},
  commentstyle=\color{gray}\itshape,
  stringstyle=\color{darkgreen}\bfseries,
  morestring=[b]",
  numberstyle=\color{teal}\bfseries,
  showstringspaces=false,
  breaklines=true,
  frame=none,
  literate=
    % Numbers
    {0}{{{\color{teal}\bfseries 0}}}1
    {1}{{{\color{teal}\bfseries 1}}}1
    {2}{{{\color{teal}\bfseries 2}}}1
    {3}{{{\color{teal}\bfseries 3}}}1
    {4}{{{\color{teal}\bfseries 4}}}1
    {5}{{{\color{teal}\bfseries 5}}}1
    {6}{{{\color{teal}\bfseries 6}}}1
    {7}{{{\color{teal}\bfseries 7}}}1
    {8}{{{\color{teal}\bfseries 8}}}1
    {9}{{{\color{teal}\bfseries 9}}}1
    % Special characters
    {:}{{{\color{darkred}\bfseries :}}}1
    {,}{{{\color{darkred}\bfseries ,}}}1
    {\{}{{{\color{darkgreen}\bfseries \{}}}1
    {\}}{{{\color{darkgreen}\bfseries \}}}}1
    {[}{{{\color{darkgreen}\bfseries [}}}1
    {]}{{{\color{darkgreen}\bfseries ]}}}1
}

% Set default listing style with enhanced appearance
\lstset{
  basicstyle=\ttfamily\footnotesize,
  numbers=none,
  numberstyle=\tiny\color{gray},
  stepnumber=1,
  numbersep=8pt,
  backgroundcolor=\color{lightgray!10},
  showspaces=false,
  showstringspaces=false,
  showtabs=false,
  frame=single,
  framerule=0.5pt,
  rulecolor=\color{darkgray},
  tabsize=2,
  captionpos=b,
  breaklines=true,
  breakatwhitespace=true,
  prebreak=\raisebox{0ex}[0ex][0ex]{\ensuremath{\hookleftarrow}},
  postbreak=\raisebox{0ex}[0ex][0ex]{\ensuremath{\hookrightarrow\space}},
  title=\lstname,
  escapeinside={\%*}{*)},
  keywordstyle=\color{blue}\bfseries,
  commentstyle=\color{gray}\itshape,
  stringstyle=\color{orange},
  xleftmargin=\parindent,
  xrightmargin=5pt,
  framexleftmargin=\parindent,
  framexrightmargin=5pt,
  columns=flexible,
  keepspaces=true
} % Phần này cho phép chèn code và formatting code như C, C++, Python
% \newgeometry{top=2cm, bottom=2cm, left=2cm, right=2cm}
\subfile{Cover} % Phần bìa
% \restoregeometry

% ===================================================
\pagestyle{empty} % Header và footer rỗng
%\newpage
%\pagenumbering{gobble} % Xóa page numbering ở cuối trang
%\subfile{chapters/0_1_subject.tex}

% \pagestyle{empty} % Header và footer rỗng
\newpage
\pagenumbering{gobble} % Xóa page numbering ở cuối trang
\subfile{Chapter/0_2_Acknowledgment.tex}

% \pagestyle{empty} % Header và footer rỗng
\newpage
\pagenumbering{gobble} % Xóa page numbering ở cuối trang
\subfile{Chapter/0_3_Abstract.tex}


% ===================================================
% \pagestyle{empty} % Header và footer rỗng
\newpage
\pagenumbering{gobble} % Xóa page numbering ở cuối trang
\renewcommand*\contentsname{TABLE OF CONTENTS}
\titlecontents{chapter}
    [0.0cm]             % left margin
    {\bfseries\vspace{0.3cm}}                  % above code
    {{\bfseries{\scshape} CHAPTER \thecontentslabel.\ }} % numbered format
    {}         % unnumbered format
    {\titlerule*[0.3pc]{.}\contentspage}         % filler-page-format, e.g dots
    
\titlecontents{section}
    [0.0cm]             % left margin
    {\vspace{0.3cm}}                  % above code
    {\thecontentslabel \ } % numbered format
    {}         % unnumbered format
    {\titlerule*[0.3pc]{.}\contentspage}         % filler-page-format, e.g dots
    
\titlecontents{subsection}
    [1.0cm]             % left margin
    {\vspace{0.3cm}}                  % above code
    {\thecontentslabel \ } % numbered format
    {}         % unnumbered format
    {\titlerule*[0.3pc]{.}\contentspage}         % filler-page-format, e.g dots

 % Tạo mục lục tự động
\addtocontents{toc}{\protect\thispagestyle{empty}}
\tableofcontents 
\thispagestyle{empty}
\cleardoublepage

\pagenumbering{roman}
%Tạo danh mục hình vẽ.
\renewcommand{\listfigurename}{LIST OF FIGURES}
{\let\oldnumberline\numberline
\renewcommand{\numberline}{Figure~\oldnumberline}
\listoffigures} 
% \phantomsection\addcontentsline{toc}{section}{\numberline {} DANH MỤC HÌNH VẼ}
\newpage


 %Tạo danh mục bảng biểu.
\renewcommand{\listtablename}{LIST OF TABLES}
{\let\oldnumberline\numberline
\renewcommand{\numberline}{Table~\oldnumberline}
\listoftables}
% \phantomsection\addcontentsline{toc}{section}{\numberline {} DANH MỤC BẢNG BIỂU}

\glsaddall 
% \renewcommand*{\glossaryname}{Danh sách thuật ngữ}
\renewcommand*{\acronymname}{LIST OF ABBREVIATIONS}
\renewcommand*{\entryname}{Abbreviation}
\renewcommand*{\descriptionname}{Definition}
\printnoidxglossaries
% \phantomsection\addcontentsline{toc}{section}{\numberline {} DANH MỤC THUẬT NGỮ VÀ TỪ VIẾT TẮT}

\renewcommand\appendixname{APPENDIX}
\renewcommand\appendixpagename{APPENDIX}
\renewcommand\appendixtocname{APPENDIX}

\renewcommand{\figurename}{Figure}
\renewcommand{\tablename}{Table}
\renewcommand{\chaptername}{CHAPTER}

% ===================================================


\newpage
\pagenumbering{arabic}

\pagestyle{fancy}
\fancyhf{}
\fancyhead[RE, LO]{\leftmark}
%\fancyhead[LE]{\rightmark}
\fancyfoot[LO]{\thepage}

\chapter{INTRODUCTION}
\label{chapter:intro}
\subfile{Chapter/1_Introduction} % Phần mở đầu

\newpage
%\pagestyle{fancy} % Áp dụng header và footer
\chapter{LITERATURE REVIEW}
\label{chapter:literature}
\subfile{Chapter/2_Literature_review}


\newpage
%\pagestyle{fancy} % Áp dụng header và footer
\chapter{METHODOLOGY}
\label{chapter:method}
\subfile{Chapter/3_0_Methodology}

\newpage
%\pagestyle{fancy} % Áp dụng header và footer
\chapter{EXPERIMENTAL RESULTS}
\label{chapter:experiment}
\subfile{Chapter/4_Experimental_results}

\newpage
\chapter{CONCLUSIONS AND FUTURE WORKS}
\label{chapter:conclusion}
\subfile{Chapter/5_Conclusions}

\newpage


% ===================================================
\newpage
\renewcommand\bibname{REFERENCE}
\printbibliography
\phantomsection\addcontentsline{toc}{chapter}{REFERENCE}

% \appendixpage
% \appendix
% \addappheadtotoc

\titleformat{\chapter}[hang]{\centering\bfseries}{ \thechapter.\ }{0pt}{}[]
\titlespacing*{\chapter}{0pt}{-20pt}{20pt}

\titlecontents{chapter}
    [0.0cm]             % left margin
    {\bfseries\vspace{0.3cm}}                  % above code
    {{\bfseries{\scshape} \thecontentslabel.\ }} % numbered format
    {}         % unnumbered format
    {\titlerule*[0.3pc]{.}\contentspage} 
    
% \chapter{Metrics Explanation}
% \subfile{Chapter/Appendix_A}

% \newpage
% \chapter{USE CASE DESCRIPTIONS}
% \subfile{Chapter/Appendix_B}

\end{document}
